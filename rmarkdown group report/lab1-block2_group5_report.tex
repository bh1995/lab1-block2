\documentclass[]{article}
\usepackage{lmodern}
\usepackage{amssymb,amsmath}
\usepackage{ifxetex,ifluatex}
\usepackage{fixltx2e} % provides \textsubscript
\ifnum 0\ifxetex 1\fi\ifluatex 1\fi=0 % if pdftex
  \usepackage[T1]{fontenc}
  \usepackage[utf8]{inputenc}
\else % if luatex or xelatex
  \ifxetex
    \usepackage{mathspec}
  \else
    \usepackage{fontspec}
  \fi
  \defaultfontfeatures{Ligatures=TeX,Scale=MatchLowercase}
\fi
% use upquote if available, for straight quotes in verbatim environments
\IfFileExists{upquote.sty}{\usepackage{upquote}}{}
% use microtype if available
\IfFileExists{microtype.sty}{%
\usepackage{microtype}
\UseMicrotypeSet[protrusion]{basicmath} % disable protrusion for tt fonts
}{}
\usepackage[margin=1in]{geometry}
\usepackage{hyperref}
\hypersetup{unicode=true,
            pdftitle={lab1 block 2 group 5 report},
            pdfauthor={Ahmed Alhasan, Erik Anders, Bjorn\_Hansen},
            pdfborder={0 0 0},
            breaklinks=true}
\urlstyle{same}  % don't use monospace font for urls
\usepackage{color}
\usepackage{fancyvrb}
\newcommand{\VerbBar}{|}
\newcommand{\VERB}{\Verb[commandchars=\\\{\}]}
\DefineVerbatimEnvironment{Highlighting}{Verbatim}{commandchars=\\\{\}}
% Add ',fontsize=\small' for more characters per line
\usepackage{framed}
\definecolor{shadecolor}{RGB}{248,248,248}
\newenvironment{Shaded}{\begin{snugshade}}{\end{snugshade}}
\newcommand{\AlertTok}[1]{\textcolor[rgb]{0.94,0.16,0.16}{#1}}
\newcommand{\AnnotationTok}[1]{\textcolor[rgb]{0.56,0.35,0.01}{\textbf{\textit{#1}}}}
\newcommand{\AttributeTok}[1]{\textcolor[rgb]{0.77,0.63,0.00}{#1}}
\newcommand{\BaseNTok}[1]{\textcolor[rgb]{0.00,0.00,0.81}{#1}}
\newcommand{\BuiltInTok}[1]{#1}
\newcommand{\CharTok}[1]{\textcolor[rgb]{0.31,0.60,0.02}{#1}}
\newcommand{\CommentTok}[1]{\textcolor[rgb]{0.56,0.35,0.01}{\textit{#1}}}
\newcommand{\CommentVarTok}[1]{\textcolor[rgb]{0.56,0.35,0.01}{\textbf{\textit{#1}}}}
\newcommand{\ConstantTok}[1]{\textcolor[rgb]{0.00,0.00,0.00}{#1}}
\newcommand{\ControlFlowTok}[1]{\textcolor[rgb]{0.13,0.29,0.53}{\textbf{#1}}}
\newcommand{\DataTypeTok}[1]{\textcolor[rgb]{0.13,0.29,0.53}{#1}}
\newcommand{\DecValTok}[1]{\textcolor[rgb]{0.00,0.00,0.81}{#1}}
\newcommand{\DocumentationTok}[1]{\textcolor[rgb]{0.56,0.35,0.01}{\textbf{\textit{#1}}}}
\newcommand{\ErrorTok}[1]{\textcolor[rgb]{0.64,0.00,0.00}{\textbf{#1}}}
\newcommand{\ExtensionTok}[1]{#1}
\newcommand{\FloatTok}[1]{\textcolor[rgb]{0.00,0.00,0.81}{#1}}
\newcommand{\FunctionTok}[1]{\textcolor[rgb]{0.00,0.00,0.00}{#1}}
\newcommand{\ImportTok}[1]{#1}
\newcommand{\InformationTok}[1]{\textcolor[rgb]{0.56,0.35,0.01}{\textbf{\textit{#1}}}}
\newcommand{\KeywordTok}[1]{\textcolor[rgb]{0.13,0.29,0.53}{\textbf{#1}}}
\newcommand{\NormalTok}[1]{#1}
\newcommand{\OperatorTok}[1]{\textcolor[rgb]{0.81,0.36,0.00}{\textbf{#1}}}
\newcommand{\OtherTok}[1]{\textcolor[rgb]{0.56,0.35,0.01}{#1}}
\newcommand{\PreprocessorTok}[1]{\textcolor[rgb]{0.56,0.35,0.01}{\textit{#1}}}
\newcommand{\RegionMarkerTok}[1]{#1}
\newcommand{\SpecialCharTok}[1]{\textcolor[rgb]{0.00,0.00,0.00}{#1}}
\newcommand{\SpecialStringTok}[1]{\textcolor[rgb]{0.31,0.60,0.02}{#1}}
\newcommand{\StringTok}[1]{\textcolor[rgb]{0.31,0.60,0.02}{#1}}
\newcommand{\VariableTok}[1]{\textcolor[rgb]{0.00,0.00,0.00}{#1}}
\newcommand{\VerbatimStringTok}[1]{\textcolor[rgb]{0.31,0.60,0.02}{#1}}
\newcommand{\WarningTok}[1]{\textcolor[rgb]{0.56,0.35,0.01}{\textbf{\textit{#1}}}}
\usepackage{graphicx}
% grffile has become a legacy package: https://ctan.org/pkg/grffile
\IfFileExists{grffile.sty}{%
\usepackage{grffile}
}{}
\makeatletter
\def\maxwidth{\ifdim\Gin@nat@width>\linewidth\linewidth\else\Gin@nat@width\fi}
\def\maxheight{\ifdim\Gin@nat@height>\textheight\textheight\else\Gin@nat@height\fi}
\makeatother
% Scale images if necessary, so that they will not overflow the page
% margins by default, and it is still possible to overwrite the defaults
% using explicit options in \includegraphics[width, height, ...]{}
\setkeys{Gin}{width=\maxwidth,height=\maxheight,keepaspectratio}
\IfFileExists{parskip.sty}{%
\usepackage{parskip}
}{% else
\setlength{\parindent}{0pt}
\setlength{\parskip}{6pt plus 2pt minus 1pt}
}
\setlength{\emergencystretch}{3em}  % prevent overfull lines
\providecommand{\tightlist}{%
  \setlength{\itemsep}{0pt}\setlength{\parskip}{0pt}}
\setcounter{secnumdepth}{0}
% Redefines (sub)paragraphs to behave more like sections
\ifx\paragraph\undefined\else
\let\oldparagraph\paragraph
\renewcommand{\paragraph}[1]{\oldparagraph{#1}\mbox{}}
\fi
\ifx\subparagraph\undefined\else
\let\oldsubparagraph\subparagraph
\renewcommand{\subparagraph}[1]{\oldsubparagraph{#1}\mbox{}}
\fi

%%% Use protect on footnotes to avoid problems with footnotes in titles
\let\rmarkdownfootnote\footnote%
\def\footnote{\protect\rmarkdownfootnote}

%%% Change title format to be more compact
\usepackage{titling}

% Create subtitle command for use in maketitle
\providecommand{\subtitle}[1]{
  \posttitle{
    \begin{center}\large#1\end{center}
    }
}

\setlength{\droptitle}{-2em}

  \title{lab1 block 2 group 5 report}
    \pretitle{\vspace{\droptitle}\centering\huge}
  \posttitle{\par}
    \author{Ahmed Alhasan, Erik Anders, Bjorn\_Hansen}
    \preauthor{\centering\large\emph}
  \postauthor{\par}
      \predate{\centering\large\emph}
  \postdate{\par}
    \date{12/3/2019}


\begin{document}
\maketitle

\hypertarget{ensemble-methods}{%
\subsection{1. ENSEMBLE METHODS}\label{ensemble-methods}}

To start, the data is loaded into test and train sets with 2/3 of the
data being used for training and 1/3 used for testing.

\begin{verbatim}
## Warning in RNGkind("Mersenne-Twister", "Inversion", "Rounding"): non-uniform
## 'Rounding' sampler used
\end{verbatim}

\hypertarget{adaboost}{%
\subsection{Adaboost}\label{adaboost}}

Adaboost classification trees are used to train the model.Inside the
boost control argument a value of 0.6 for ``nu'' was used as this value
gave the lowest error. Adaboost predicts a loss score so to make a hard
decision a threshold of 0.1 was used as this value gave roughly the
lowest error. As can be seen from the confusion matrix below (using 100
trees) the diagonal values are quite high meaning that the ada model is
predicting with high accuaracy.

\begin{verbatim}
## Loading required package: parallel
\end{verbatim}

\begin{verbatim}
## Loading required package: stabs
\end{verbatim}

\begin{verbatim}
## This is mboost 2.9-1. See 'package?mboost' and 'news(package  = "mboost")'
## for a complete list of changes.
\end{verbatim}

\begin{verbatim}
##    Y hat
## Y     0   1
##   0 893  29
##   1  51 561
\end{verbatim}

\begin{verbatim}
## [1] "Error for ada model: 0.0521512385919165"
\end{verbatim}

Below is a plot of the recorded error for 10 up to 100 trees.

\begin{verbatim}
## 
## Attaching package: 'ggplot2'
\end{verbatim}

\begin{verbatim}
## The following object is masked from 'package:mboost':
## 
##     %+%
\end{verbatim}

\includegraphics{lab1-block2_group5_report_files/figure-latex/unnamed-chunk-3-1.pdf}

\hypertarget{random-forest}{%
\subsection{Random Forest}\label{random-forest}}

The random forest algorithm is used to classify the mails as spam and
non-spam. As can be seen from the confusion matrix below (using 500
trees) the diagonal values are very high meaning that the randomForest
model is predicting with high accuaracy.

\begin{verbatim}
## randomForest 4.6-14
\end{verbatim}

\begin{verbatim}
## Type rfNews() to see new features/changes/bug fixes.
\end{verbatim}

\begin{verbatim}
## 
## Attaching package: 'randomForest'
\end{verbatim}

\begin{verbatim}
## The following object is masked from 'package:ggplot2':
## 
##     margin
\end{verbatim}

\begin{verbatim}
##    Y hat
## Y     0   1
##   0 888  34
##   1  39 573
\end{verbatim}

\begin{verbatim}
## [1] 0.04758801
\end{verbatim}

Below is a plot of the recorded error for 10 up to 100 trees.

\includegraphics{lab1-block2_group5_report_files/figure-latex/unnamed-chunk-5-1.pdf}

\hypertarget{mixture-models}{%
\subsection{2. Mixture Models}\label{mixture-models}}

\hypertarget{analysis}{%
\subsection{Analysis:}\label{analysis}}

\begin{itemize}
\tightlist
\item
  After we created our data points using parameters mu and pi, the EM
  function takes only the data points and guess mu and pi(pi here is
  like a prior probablity based on our best guess that the
  responsiblities are equal but it can be set differently)
\item
  In E-step we compute the posterior responsiblities for each
  observation based on Bayes Theorem, if K=2 we make the assumbtion that
  the points are in two clusters, anf if it is 3 the we assume 3
  clustersand so on. γ(znk) = πk ∗ p(xn\textbar{}µk) PK j=1 πj ∗
  p(xn\textbar{}µj ) , where i = 1, 2, \ldots{}, N
\item
  In the M-step we updated our mu and pi based on the new
  responsiblities. µnew k = 1/Nk X N n=1 γ(znk)xn π new k = Nk N
\item
  Log likelikelihood then is calculated (doesn't matter if it is done
  before M-step or after) to check if the updated values of mu and pi
  are converging to the true values.
\item
  In the case of K=2 little convergence gained after the 8th iteration
  and stopped at 12th iteration when we started to get minimum change.
  the resulted values where close to the true µ1 and µ2, this because µ3
  value was 0.5 in the middle of the other clusters, so this third
  cluster splitted between the first two clusters without complications.
\item
  When K=3 the convergence also start to blateau when we reached the 8th
  iteration and stopped at 46th iteration, this is because the values of
  first two true mus overlapping the the third mu, making it complicated
  for the algorithm to distinguish the third cluster from the other two
  (which are more distinct from each other).
\item
  The same thing happened when K=4 also the convergence also start to
  blateau when we reached the 8th iteration and stopped at 32nd
  iteration this time, this is because the third cluster around true Mu3
  got split into two cluster, one that is close to the first Mu and the
  other close to the second Mu.
\item
  We can conclude from fact that the algorithm start to gain little
  convergence at the 8th iteration is because the first two original
  clusters are more distinguishable from the third cluster because their
  Mus overlapp the third mu, and once the estimated Mus for these two
  clusters are gained, the algorithm find it difficult to recognize the
  third cluster.
  \includegraphics{lab1-block2_group5_report_files/figure-latex/unnamed-chunk-6-1.pdf}
\end{itemize}

\begin{verbatim}
## iteration:  1 log likelihood:  -6930.975 
## iteration:  2 log likelihood:  -6929.125 
## iteration:  3 log likelihood:  -6928.562 
## iteration:  4 log likelihood:  -6924.281 
## iteration:  5 log likelihood:  -6893.055 
## iteration:  6 log likelihood:  -6728.948 
## iteration:  7 log likelihood:  -6443.28 
## iteration:  8 log likelihood:  -6368.318 
## iteration:  9 log likelihood:  -6363.734 
## iteration:  10 log likelihood:  -6363.109 
## iteration:  11 log likelihood:  -6362.947 
## iteration:  12 log likelihood:  -6362.897
\end{verbatim}

\includegraphics{lab1-block2_group5_report_files/figure-latex/unnamed-chunk-6-2.pdf}
\includegraphics{lab1-block2_group5_report_files/figure-latex/unnamed-chunk-6-3.pdf}

\begin{verbatim}
## iteration:  1 log likelihood:  -6931.064 
## iteration:  2 log likelihood:  -6928.051 
## iteration:  3 log likelihood:  -6920.026 
## iteration:  4 log likelihood:  -6864.176 
## iteration:  5 log likelihood:  -6634.916 
## iteration:  6 log likelihood:  -6409.234 
## iteration:  7 log likelihood:  -6373.593 
## iteration:  8 log likelihood:  -6367.833 
## iteration:  9 log likelihood:  -6364.983 
## iteration:  10 log likelihood:  -6363.074 
## iteration:  11 log likelihood:  -6361.594 
## iteration:  12 log likelihood:  -6360.309 
## iteration:  13 log likelihood:  -6359.103 
## iteration:  14 log likelihood:  -6357.93 
## iteration:  15 log likelihood:  -6356.786 
## iteration:  16 log likelihood:  -6355.689 
## iteration:  17 log likelihood:  -6354.668 
## iteration:  18 log likelihood:  -6353.742 
## iteration:  19 log likelihood:  -6352.92 
## iteration:  20 log likelihood:  -6352.199 
## iteration:  21 log likelihood:  -6351.567 
## iteration:  22 log likelihood:  -6351.011 
## iteration:  23 log likelihood:  -6350.515 
## iteration:  24 log likelihood:  -6350.069 
## iteration:  25 log likelihood:  -6349.661 
## iteration:  26 log likelihood:  -6349.286 
## iteration:  27 log likelihood:  -6348.938 
## iteration:  28 log likelihood:  -6348.616 
## iteration:  29 log likelihood:  -6348.315 
## iteration:  30 log likelihood:  -6348.036 
## iteration:  31 log likelihood:  -6347.776 
## iteration:  32 log likelihood:  -6347.534 
## iteration:  33 log likelihood:  -6347.308 
## iteration:  34 log likelihood:  -6347.099 
## iteration:  35 log likelihood:  -6346.904 
## iteration:  36 log likelihood:  -6346.722 
## iteration:  37 log likelihood:  -6346.553 
## iteration:  38 log likelihood:  -6346.394 
## iteration:  39 log likelihood:  -6346.246 
## iteration:  40 log likelihood:  -6346.107 
## iteration:  41 log likelihood:  -6345.977 
## iteration:  42 log likelihood:  -6345.854 
## iteration:  43 log likelihood:  -6345.739 
## iteration:  44 log likelihood:  -6345.63 
## iteration:  45 log likelihood:  -6345.528 
## iteration:  46 log likelihood:  -6345.431
\end{verbatim}

\includegraphics{lab1-block2_group5_report_files/figure-latex/unnamed-chunk-6-4.pdf}
\includegraphics{lab1-block2_group5_report_files/figure-latex/unnamed-chunk-6-5.pdf}

\begin{verbatim}
## iteration:  1 log likelihood:  -6930.838 
## iteration:  2 log likelihood:  -6928.641 
## iteration:  3 log likelihood:  -6924.748 
## iteration:  4 log likelihood:  -6896.25 
## iteration:  5 log likelihood:  -6741.896 
## iteration:  6 log likelihood:  -6452.658 
## iteration:  7 log likelihood:  -6366.493 
## iteration:  8 log likelihood:  -6359.764 
## iteration:  9 log likelihood:  -6357.876 
## iteration:  10 log likelihood:  -6356.372 
## iteration:  11 log likelihood:  -6354.86 
## iteration:  12 log likelihood:  -6353.31 
## iteration:  13 log likelihood:  -6351.776 
## iteration:  14 log likelihood:  -6350.33 
## iteration:  15 log likelihood:  -6349.03 
## iteration:  16 log likelihood:  -6347.908 
## iteration:  17 log likelihood:  -6346.968 
## iteration:  18 log likelihood:  -6346.196 
## iteration:  19 log likelihood:  -6345.566 
## iteration:  20 log likelihood:  -6345.055 
## iteration:  21 log likelihood:  -6344.637 
## iteration:  22 log likelihood:  -6344.293 
## iteration:  23 log likelihood:  -6344.008 
## iteration:  24 log likelihood:  -6343.768 
## iteration:  25 log likelihood:  -6343.563 
## iteration:  26 log likelihood:  -6343.387 
## iteration:  27 log likelihood:  -6343.233 
## iteration:  28 log likelihood:  -6343.097 
## iteration:  29 log likelihood:  -6342.975 
## iteration:  30 log likelihood:  -6342.864 
## iteration:  31 log likelihood:  -6342.762 
## iteration:  32 log likelihood:  -6342.668
\end{verbatim}

\includegraphics{lab1-block2_group5_report_files/figure-latex/unnamed-chunk-6-6.pdf}
\includegraphics{lab1-block2_group5_report_files/figure-latex/unnamed-chunk-6-7.pdf}

\#\#Apendix

\begin{Shaded}
\begin{Highlighting}[]
\KeywordTok{library}\NormalTok{(mboost)}
\KeywordTok{library}\NormalTok{(readxl)}
\KeywordTok{library}\NormalTok{(randomForest)}
\KeywordTok{library}\NormalTok{(ggplot2)}

\CommentTok{#setwd("C:/Users/Bjorn/Documents/LIU/machine_learning")}
\CommentTok{#data=read_excel("spambase.xlsx")}

\NormalTok{sp =}\StringTok{ }\KeywordTok{read.csv2}\NormalTok{(}\StringTok{"spambase_lab1_block2.csv"}\NormalTok{)}
\NormalTok{sp}\OperatorTok{$}\NormalTok{Spam =}\StringTok{ }\KeywordTok{as.factor}\NormalTok{(sp}\OperatorTok{$}\NormalTok{Spam)}

\NormalTok{n=}\KeywordTok{dim}\NormalTok{(sp)[}\DecValTok{1}\NormalTok{]}
\KeywordTok{set.seed}\NormalTok{(}\DecValTok{12345}\NormalTok{)}
\NormalTok{id=}\KeywordTok{sample}\NormalTok{(}\DecValTok{1}\OperatorTok{:}\NormalTok{n, }\KeywordTok{floor}\NormalTok{(n}\OperatorTok{*}\NormalTok{(}\DecValTok{2}\OperatorTok{/}\DecValTok{3}\NormalTok{)))}
\NormalTok{train=sp[id,]}
\NormalTok{test=sp[}\OperatorTok{-}\NormalTok{id,]}

\CommentTok{## Adaboost ##}

\NormalTok{ada_model =}\StringTok{ }\KeywordTok{blackboost}\NormalTok{(Spam }\OperatorTok{~}\NormalTok{., }\DataTypeTok{data =}\NormalTok{ train, }\DataTypeTok{control =} \KeywordTok{boost_control}\NormalTok{(}\DataTypeTok{mstop =} \DecValTok{100}\NormalTok{, }\DataTypeTok{nu=}\FloatTok{0.6}\NormalTok{),}
                          \DataTypeTok{family =} \KeywordTok{AdaExp}\NormalTok{())}

\NormalTok{acc=}\DecValTok{0}
\NormalTok{n =}\StringTok{ }\KeywordTok{seq}\NormalTok{(}\DecValTok{10}\NormalTok{, }\DecValTok{100}\NormalTok{, }\DataTypeTok{by =} \DecValTok{10}\NormalTok{)}
\ControlFlowTok{for}\NormalTok{(i }\ControlFlowTok{in}\NormalTok{ n)\{}
\NormalTok{  ada_model =}\StringTok{ }\KeywordTok{blackboost}\NormalTok{(Spam }\OperatorTok{~}\NormalTok{., }\DataTypeTok{data =}\NormalTok{ train, }\DataTypeTok{control =} \KeywordTok{boost_control}\NormalTok{(}\DataTypeTok{mstop =}\NormalTok{ i, }\DataTypeTok{nu =} \FloatTok{0.6}\NormalTok{),}
                       \DataTypeTok{family =} \KeywordTok{AdaExp}\NormalTok{())}
\NormalTok{  fitted.results_test=}\StringTok{ }\KeywordTok{predict}\NormalTok{(ada_model,}\DataTypeTok{newdata=}\NormalTok{test)}
\NormalTok{  fitted.results_test =}\StringTok{ }\KeywordTok{ifelse}\NormalTok{(fitted.results_test }\OperatorTok{>}\StringTok{ }\FloatTok{0.1}\NormalTok{,}\DecValTok{1}\NormalTok{,}\DecValTok{0}\NormalTok{)}
\NormalTok{  misClasificError_test =}\StringTok{ }\KeywordTok{mean}\NormalTok{(fitted.results_test }\OperatorTok{!=}\StringTok{ }\NormalTok{test}\OperatorTok{$}\NormalTok{Spam)}
\NormalTok{  ada_confmat_test =}\StringTok{ }\KeywordTok{table}\NormalTok{(}\StringTok{"Y"}\NormalTok{=test}\OperatorTok{$}\NormalTok{Spam,}\StringTok{"Y hat"}\NormalTok{=fitted.results_test)}
\NormalTok{  acc[i] =}\StringTok{ }\KeywordTok{sum}\NormalTok{(}\KeywordTok{diag}\NormalTok{(ada_confmat_test)) }\OperatorTok{/}\StringTok{ }\KeywordTok{sum}\NormalTok{(ada_confmat_test)}
\NormalTok{\} }
\NormalTok{acc}

\NormalTok{y_error <-}\StringTok{ }\KeywordTok{na.omit}\NormalTok{(}\DecValTok{1}\OperatorTok{-}\NormalTok{acc)}
\NormalTok{y_error =}\StringTok{ }\NormalTok{y_error[}\OperatorTok{-}\DecValTok{1}\NormalTok{]}
\NormalTok{x_trees <-}\StringTok{ }\NormalTok{n}
\NormalTok{ada_m <-}\StringTok{ }\KeywordTok{na.omit}\NormalTok{(}\KeywordTok{data.frame}\NormalTok{(x_trees, y_error))}

\KeywordTok{ggplot}\NormalTok{()}\OperatorTok{+}
\StringTok{  }\KeywordTok{ggtitle}\NormalTok{(}\StringTok{" Error for Adaboost "}\NormalTok{)}\OperatorTok{+}
\StringTok{  }\KeywordTok{geom_point}\NormalTok{(}\DataTypeTok{data=}\NormalTok{ada_m, }\KeywordTok{aes}\NormalTok{(}\DataTypeTok{x=}\NormalTok{x_trees, }\DataTypeTok{y=}\NormalTok{y_error), }\DataTypeTok{size=}\DecValTok{2}\NormalTok{)}\OperatorTok{+}
\StringTok{  }\KeywordTok{geom_line}\NormalTok{(}\DataTypeTok{data=}\NormalTok{ada_m, }\KeywordTok{aes}\NormalTok{(}\DataTypeTok{x=}\NormalTok{x_trees, }\DataTypeTok{y=}\NormalTok{y_error), }\DataTypeTok{size=}\DecValTok{1}\NormalTok{)}


\CommentTok{#train data}
\NormalTok{fitted.results_train =}\StringTok{ }\KeywordTok{predict}\NormalTok{(ada_model,}\DataTypeTok{newdata=}\NormalTok{train)}
\NormalTok{fitted.results_train =}\StringTok{ }\KeywordTok{ifelse}\NormalTok{(fitted.results_train }\OperatorTok{>}\StringTok{ }\FloatTok{0.1}\NormalTok{,}\DecValTok{1}\NormalTok{,}\DecValTok{0}\NormalTok{)}
\NormalTok{misClasificError_train =}\StringTok{ }\KeywordTok{mean}\NormalTok{(fitted.results_train }\OperatorTok{!=}\StringTok{ }\NormalTok{train}\OperatorTok{$}\NormalTok{Spam)}
\NormalTok{ada_confmat_train =}\StringTok{ }\KeywordTok{table}\NormalTok{(}\StringTok{"Y"}\NormalTok{=train}\OperatorTok{$}\NormalTok{Spam,}\StringTok{"Y hat"}\NormalTok{=fitted.results_train)}
\NormalTok{ada_confmat_train}
\KeywordTok{print}\NormalTok{(}\KeywordTok{paste}\NormalTok{(}\StringTok{'Accuracy:'}\NormalTok{,}\KeywordTok{sum}\NormalTok{(}\KeywordTok{diag}\NormalTok{(ada_confmat_train)) }\OperatorTok{/}\StringTok{ }\KeywordTok{sum}\NormalTok{(ada_confmat_train)))}

\CommentTok{#test data}
\NormalTok{fitted.results_test=}\StringTok{ }\KeywordTok{predict}\NormalTok{(ada_model,}\DataTypeTok{newdata=}\NormalTok{test)}
\NormalTok{fitted.results_test =}\StringTok{ }\KeywordTok{ifelse}\NormalTok{(fitted.results_test }\OperatorTok{>}\StringTok{ }\FloatTok{0.1}\NormalTok{,}\DecValTok{1}\NormalTok{,}\DecValTok{0}\NormalTok{)}
\NormalTok{misClasificError_test =}\StringTok{ }\KeywordTok{mean}\NormalTok{(fitted.results_test }\OperatorTok{!=}\StringTok{ }\NormalTok{test}\OperatorTok{$}\NormalTok{Spam)}
\NormalTok{ada_confmat_test =}\StringTok{ }\KeywordTok{table}\NormalTok{(}\StringTok{"Y"}\NormalTok{=test}\OperatorTok{$}\NormalTok{Spam,}\StringTok{"Y hat"}\NormalTok{=fitted.results_test)}
\KeywordTok{print}\NormalTok{(}\KeywordTok{paste}\NormalTok{(}\StringTok{'Accuracy:'}\NormalTok{,}\KeywordTok{sum}\NormalTok{(}\KeywordTok{diag}\NormalTok{(ada_confmat_test)) }\OperatorTok{/}\StringTok{ }\KeywordTok{sum}\NormalTok{(ada_confmat_test)))}
\CommentTok{#print(paste('Accuracy:',1-misClasificError_test))}

\CommentTok{## Random Forest ##}

\NormalTok{rf<-}\KeywordTok{randomForest}\NormalTok{(Spam }\OperatorTok{~}\StringTok{ }\NormalTok{., }\DataTypeTok{data=}\NormalTok{ train)}
\NormalTok{  p <-}\StringTok{ }\KeywordTok{predict}\NormalTok{(rf, }\DataTypeTok{newdata =}\NormalTok{ test, }\DataTypeTok{type =} \StringTok{"class"}\NormalTok{)}
  \KeywordTok{print}\NormalTok{(}\KeywordTok{table}\NormalTok{(}\StringTok{"Y"}\NormalTok{=test}\OperatorTok{$}\NormalTok{Spam,}\StringTok{"Y hat"}\NormalTok{=p))}
\NormalTok{  misClasificError <-}\StringTok{ }\KeywordTok{mean}\NormalTok{(p }\OperatorTok{!=}\StringTok{ }\NormalTok{test}\OperatorTok{$}\NormalTok{Spam)}
\NormalTok{  misClasificError}
  
\NormalTok{misClasificError <-}\StringTok{ }\DecValTok{0}
\ControlFlowTok{for}\NormalTok{ (i }\ControlFlowTok{in} \KeywordTok{seq}\NormalTok{(}\DecValTok{10}\NormalTok{, }\DecValTok{100}\NormalTok{, }\DecValTok{10}\NormalTok{)) \{}
\NormalTok{  rf<-}\KeywordTok{randomForest}\NormalTok{(Spam }\OperatorTok{~}\StringTok{ }\NormalTok{., }\DataTypeTok{data=}\NormalTok{ train, }\DataTypeTok{ntree=}\NormalTok{i)}
\NormalTok{  p <-}\StringTok{ }\KeywordTok{predict}\NormalTok{(rf, }\DataTypeTok{newdata =}\NormalTok{ test, }\DataTypeTok{type =} \StringTok{"class"}\NormalTok{)}
\NormalTok{  misClasificError[(i }\OperatorTok{/}\StringTok{ }\DecValTok{10}\NormalTok{)] <-}\StringTok{ }\KeywordTok{mean}\NormalTok{(p }\OperatorTok{!=}\StringTok{ }\NormalTok{test}\OperatorTok{$}\NormalTok{Spam)}
\NormalTok{\}}
\NormalTok{rf_df<-}\KeywordTok{data.frame}\NormalTok{(}\StringTok{"rferror"}\NormalTok{=misClasificError,}\StringTok{"rftrees"}\NormalTok{=}\KeywordTok{seq}\NormalTok{(}\DecValTok{10}\NormalTok{,}\DecValTok{100}\NormalTok{,}\DecValTok{10}\NormalTok{))}

\KeywordTok{ggplot}\NormalTok{()}\OperatorTok{+}
\StringTok{  }\KeywordTok{ggtitle}\NormalTok{(}\StringTok{" Error for RandomForrest "}\NormalTok{)}\OperatorTok{+}
\StringTok{  }\KeywordTok{geom_point}\NormalTok{(}\DataTypeTok{data=}\NormalTok{rf_df, }\KeywordTok{aes}\NormalTok{(}\DataTypeTok{x=}\NormalTok{rftrees, }\DataTypeTok{y=}\NormalTok{rferror), }\DataTypeTok{size=}\DecValTok{2}\NormalTok{)}\OperatorTok{+}
\StringTok{  }\KeywordTok{geom_line}\NormalTok{(}\DataTypeTok{data=}\NormalTok{rf_df, }\KeywordTok{aes}\NormalTok{(}\DataTypeTok{x=}\NormalTok{rftrees, }\DataTypeTok{y=}\NormalTok{rferror), }\DataTypeTok{size=}\DecValTok{1}\NormalTok{)}


\CommentTok{## Mixed models / EM algorithm ##}

\KeywordTok{set.seed}\NormalTok{(}\DecValTok{1234567890}\NormalTok{)}
\NormalTok{max_it <-}\StringTok{ }\DecValTok{100} \CommentTok{# max number of EM iterations}
\NormalTok{min_change <-}\StringTok{ }\FloatTok{0.1} \CommentTok{# min change in log likelihood between two consecutive EM iterations}
\NormalTok{N=}\DecValTok{1000} \CommentTok{# number of training points}
\NormalTok{D=}\DecValTok{10} \CommentTok{# number of dimensions}
\NormalTok{x <-}\StringTok{ }\KeywordTok{matrix}\NormalTok{(}\DataTypeTok{nrow=}\NormalTok{N, }\DataTypeTok{ncol=}\NormalTok{D) }\CommentTok{# training data}
\NormalTok{true_pi <-}\StringTok{ }\KeywordTok{vector}\NormalTok{(}\DataTypeTok{length =} \DecValTok{3}\NormalTok{) }\CommentTok{# true mixing coefficients}
\NormalTok{true_mu <-}\StringTok{ }\KeywordTok{matrix}\NormalTok{(}\DataTypeTok{nrow=}\DecValTok{3}\NormalTok{, }\DataTypeTok{ncol=}\NormalTok{D) }\CommentTok{# true conditional distributions}

\NormalTok{true_pi=}\KeywordTok{c}\NormalTok{(}\DecValTok{1}\OperatorTok{/}\DecValTok{3}\NormalTok{, }\DecValTok{1}\OperatorTok{/}\DecValTok{3}\NormalTok{, }\DecValTok{1}\OperatorTok{/}\DecValTok{3}\NormalTok{)}
\NormalTok{true_mu[}\DecValTok{1}\NormalTok{,]=}\KeywordTok{c}\NormalTok{(}\FloatTok{0.5}\NormalTok{,}\FloatTok{0.6}\NormalTok{,}\FloatTok{0.4}\NormalTok{,}\FloatTok{0.7}\NormalTok{,}\FloatTok{0.3}\NormalTok{,}\FloatTok{0.8}\NormalTok{,}\FloatTok{0.2}\NormalTok{,}\FloatTok{0.9}\NormalTok{,}\FloatTok{0.1}\NormalTok{,}\DecValTok{1}\NormalTok{)}
\NormalTok{true_mu[}\DecValTok{2}\NormalTok{,]=}\KeywordTok{c}\NormalTok{(}\FloatTok{0.5}\NormalTok{,}\FloatTok{0.4}\NormalTok{,}\FloatTok{0.6}\NormalTok{,}\FloatTok{0.3}\NormalTok{,}\FloatTok{0.7}\NormalTok{,}\FloatTok{0.2}\NormalTok{,}\FloatTok{0.8}\NormalTok{,}\FloatTok{0.1}\NormalTok{,}\FloatTok{0.9}\NormalTok{,}\DecValTok{0}\NormalTok{)}
\NormalTok{true_mu[}\DecValTok{3}\NormalTok{,]=}\KeywordTok{c}\NormalTok{(}\FloatTok{0.5}\NormalTok{,}\FloatTok{0.5}\NormalTok{,}\FloatTok{0.5}\NormalTok{,}\FloatTok{0.5}\NormalTok{,}\FloatTok{0.5}\NormalTok{,}\FloatTok{0.5}\NormalTok{,}\FloatTok{0.5}\NormalTok{,}\FloatTok{0.5}\NormalTok{,}\FloatTok{0.5}\NormalTok{,}\FloatTok{0.5}\NormalTok{)}
\KeywordTok{plot}\NormalTok{(true_mu[}\DecValTok{1}\NormalTok{,],}
     \DataTypeTok{type=}\StringTok{"o"}\NormalTok{,}
     \DataTypeTok{col=}\StringTok{"blue"}\NormalTok{,}
     \DataTypeTok{ylim=}\KeywordTok{c}\NormalTok{(}\DecValTok{0}\NormalTok{,}\DecValTok{1}\NormalTok{),}
     \DataTypeTok{main =} \StringTok{"Original Data"}\NormalTok{,}
     \DataTypeTok{xlab =} \StringTok{"Number of Dimensions"}\NormalTok{,}
     \DataTypeTok{ylab =} \StringTok{"True Mu"}\NormalTok{)}
\KeywordTok{points}\NormalTok{(true_mu[}\DecValTok{2}\NormalTok{,], }\DataTypeTok{type=}\StringTok{"o"}\NormalTok{, }\DataTypeTok{col=}\StringTok{"red"}\NormalTok{)}
\KeywordTok{points}\NormalTok{(true_mu[}\DecValTok{3}\NormalTok{,], }\DataTypeTok{type=}\StringTok{"o"}\NormalTok{, }\DataTypeTok{col=}\StringTok{"green"}\NormalTok{)}
\CommentTok{# Producing the training data}
\ControlFlowTok{for}\NormalTok{(n }\ControlFlowTok{in} \DecValTok{1}\OperatorTok{:}\NormalTok{N) \{}
\NormalTok{  k <-}\StringTok{ }\KeywordTok{sample}\NormalTok{(}\DecValTok{1}\OperatorTok{:}\DecValTok{3}\NormalTok{,}\DecValTok{1}\NormalTok{,}\DataTypeTok{prob=}\NormalTok{true_pi)}
  \ControlFlowTok{for}\NormalTok{(d }\ControlFlowTok{in} \DecValTok{1}\OperatorTok{:}\NormalTok{D) \{}
\NormalTok{    x[n,d] <-}\StringTok{ }\KeywordTok{rbinom}\NormalTok{(}\DecValTok{1}\NormalTok{,}\DecValTok{1}\NormalTok{,true_mu[k,d])}
\NormalTok{  \}}
\NormalTok{\}}
\NormalTok{em_algorithm <-}\StringTok{ }\ControlFlowTok{function}\NormalTok{(c)\{}
\NormalTok{  K=c }\CommentTok{# number of guessed components}
\NormalTok{  z <-}\StringTok{ }\KeywordTok{matrix}\NormalTok{(}\DataTypeTok{nrow=}\NormalTok{N, }\DataTypeTok{ncol=}\NormalTok{K) }\CommentTok{# fractional component assignments}
\NormalTok{  pi <-}\StringTok{ }\KeywordTok{vector}\NormalTok{(}\DataTypeTok{length =}\NormalTok{ K) }\CommentTok{# mixing coefficients}
\NormalTok{  mu <-}\StringTok{ }\KeywordTok{matrix}\NormalTok{(}\DataTypeTok{nrow=}\NormalTok{K, }\DataTypeTok{ncol=}\NormalTok{D) }\CommentTok{# conditional distributions}
\NormalTok{  llik <-}\StringTok{ }\KeywordTok{vector}\NormalTok{(}\DataTypeTok{length =}\NormalTok{ max_it) }\CommentTok{# log likelihood of the EM iterations}
  \CommentTok{# Random initialization of the paramters}
\NormalTok{  pi <-}\StringTok{ }\KeywordTok{runif}\NormalTok{(K,}\FloatTok{0.49}\NormalTok{,}\FloatTok{0.51}\NormalTok{)}
\NormalTok{  pi <-}\StringTok{ }\NormalTok{pi }\OperatorTok{/}\StringTok{ }\KeywordTok{sum}\NormalTok{(pi)}
  \ControlFlowTok{for}\NormalTok{(k }\ControlFlowTok{in} \DecValTok{1}\OperatorTok{:}\NormalTok{K) \{}
\NormalTok{    mu[k,] <-}\StringTok{ }\KeywordTok{runif}\NormalTok{(D,}\FloatTok{0.49}\NormalTok{,}\FloatTok{0.51}\NormalTok{)}
\NormalTok{  \}}
\NormalTok{  pi}
\NormalTok{  mu}
  \ControlFlowTok{for}\NormalTok{(it }\ControlFlowTok{in} \DecValTok{1}\OperatorTok{:}\NormalTok{max_it) \{}
    \CommentTok{#plot(mu[1,], type="o", col="blue", ylim=c(0,1))}
    \CommentTok{#points(mu[2,], type="o", col="red")}
    \CommentTok{#points(mu[3,], type="o", col="green")}
    \CommentTok{#points(mu[4,], type="o", col="yellow")}
    \CommentTok{#Sys.sleep(0.5)}
    \CommentTok{# E-step: Computation of the fractional component assignments (responsiblities)}
    \CommentTok{# Your code here}
    \ControlFlowTok{for}\NormalTok{ (n }\ControlFlowTok{in} \DecValTok{1}\OperatorTok{:}\NormalTok{N)\{}
\NormalTok{      phi =}\StringTok{ }\KeywordTok{c}\NormalTok{()}
      \ControlFlowTok{for}\NormalTok{ (j }\ControlFlowTok{in} \DecValTok{1}\OperatorTok{:}\NormalTok{K)\{}
\NormalTok{        y1 =}\StringTok{ }\NormalTok{mu[j,]}\OperatorTok{^}\NormalTok{x[n,]}
\NormalTok{        y2 =}\StringTok{ }\NormalTok{(}\DecValTok{1}\OperatorTok{-}\StringTok{ }\NormalTok{mu[j,])}\OperatorTok{^}\NormalTok{(}\DecValTok{1}\OperatorTok{-}\NormalTok{x[n,])}
\NormalTok{        phi =}\StringTok{ }\KeywordTok{c}\NormalTok{(phi, }\KeywordTok{prod}\NormalTok{(y1,y2))}
\NormalTok{      \}}
      
\NormalTok{      z[n,] =}\StringTok{ }\NormalTok{(pi}\OperatorTok{*}\NormalTok{phi) }\OperatorTok{/}\StringTok{ }\KeywordTok{sum}\NormalTok{(pi}\OperatorTok{*}\NormalTok{phi)}
\NormalTok{    \}}
    \CommentTok{#Log likelihood computation.}
    \CommentTok{# Your code here}
\NormalTok{    likelihood =}\StringTok{ }\KeywordTok{matrix}\NormalTok{(}\DecValTok{0}\NormalTok{,}\DecValTok{1000}\NormalTok{,K)}
\NormalTok{    llik[it] =}\StringTok{ }\DecValTok{0}
    \ControlFlowTok{for}\NormalTok{(n }\ControlFlowTok{in} \DecValTok{1}\OperatorTok{:}\NormalTok{N)\{}
      \ControlFlowTok{for}\NormalTok{ (k }\ControlFlowTok{in} \DecValTok{1}\OperatorTok{:}\NormalTok{K)\{}
\NormalTok{        likelihood[n,k] =}\StringTok{ }\NormalTok{pi[k]}\OperatorTok{*}\KeywordTok{prod}\NormalTok{( ((mu[k,]}\OperatorTok{^}\NormalTok{x[n,])}\OperatorTok{*}\NormalTok{((}\DecValTok{1}\OperatorTok{-}\NormalTok{mu[k,])}\OperatorTok{^}\NormalTok{(}\DecValTok{1}\OperatorTok{-}\NormalTok{x[n,]))))}
\NormalTok{      \}}
\NormalTok{      llik[it] =}\StringTok{ }\KeywordTok{sum}\NormalTok{(}\KeywordTok{log}\NormalTok{(}\KeywordTok{rowSums}\NormalTok{(likelihood)))}
\NormalTok{    \}}
    \KeywordTok{cat}\NormalTok{(}\StringTok{"iteration: "}\NormalTok{, it, }\StringTok{"log likelihood: "}\NormalTok{, llik[it], }\StringTok{"}\CharTok{\textbackslash{}n}\StringTok{"}\NormalTok{)}
    \KeywordTok{flush.console}\NormalTok{()}
    \CommentTok{# Stop if the lok likelihood has not changed significantly}
    \CommentTok{# Your code here}
    \ControlFlowTok{if}\NormalTok{ (it }\OperatorTok{>}\StringTok{ }\DecValTok{1}\NormalTok{)\{}
      \ControlFlowTok{if}\NormalTok{ (llik[it]}\OperatorTok{-}\NormalTok{llik[it}\DecValTok{-1}\NormalTok{] }\OperatorTok{<}\StringTok{ }\NormalTok{min_change)\{}
        \ControlFlowTok{if}\NormalTok{(K }\OperatorTok{==}\StringTok{ }\DecValTok{2}\NormalTok{)\{}
          \KeywordTok{plot}\NormalTok{(mu[}\DecValTok{1}\NormalTok{,],}
               \DataTypeTok{type=}\StringTok{"o"}\NormalTok{,}
               \DataTypeTok{col=}\StringTok{"blue"}\NormalTok{,}
               \DataTypeTok{ylim=}\KeywordTok{c}\NormalTok{(}\DecValTok{0}\NormalTok{,}\DecValTok{1}\NormalTok{),}
               \DataTypeTok{main =} \StringTok{"K = 2"}\NormalTok{,}
               \DataTypeTok{xlab =} \StringTok{"Number of Dimensions"}\NormalTok{,}
               \DataTypeTok{ylab =} \StringTok{"Estimated Mu"}\NormalTok{)}
          \KeywordTok{points}\NormalTok{(mu[}\DecValTok{2}\NormalTok{,], }\DataTypeTok{type=}\StringTok{"o"}\NormalTok{, }\DataTypeTok{col=}\StringTok{"red"}\NormalTok{)}
\NormalTok{        \}}
        \ControlFlowTok{else} \ControlFlowTok{if}\NormalTok{(K }\OperatorTok{==}\StringTok{ }\DecValTok{3}\NormalTok{)\{}
          \KeywordTok{plot}\NormalTok{(mu[}\DecValTok{1}\NormalTok{,],}
               \DataTypeTok{type=}\StringTok{"o"}\NormalTok{,}
               \DataTypeTok{col=}\StringTok{"blue"}\NormalTok{,}
               \DataTypeTok{ylim=}\KeywordTok{c}\NormalTok{(}\DecValTok{0}\NormalTok{,}\DecValTok{1}\NormalTok{),}
               \DataTypeTok{main =} \StringTok{"K = 3"}\NormalTok{,}
               \DataTypeTok{xlab =} \StringTok{"Number of Dimensions"}\NormalTok{,}
               \DataTypeTok{ylab =} \StringTok{"Estimated Mu"}\NormalTok{)}
          \KeywordTok{points}\NormalTok{(mu[}\DecValTok{2}\NormalTok{,], }\DataTypeTok{type=}\StringTok{"o"}\NormalTok{, }\DataTypeTok{col=}\StringTok{"red"}\NormalTok{)}
          \KeywordTok{points}\NormalTok{(mu[}\DecValTok{3}\NormalTok{,], }\DataTypeTok{type=}\StringTok{"o"}\NormalTok{, }\DataTypeTok{col=}\StringTok{"green"}\NormalTok{)}
\NormalTok{        \}}
        \ControlFlowTok{else} \ControlFlowTok{if}\NormalTok{(K }\OperatorTok{==}\StringTok{ }\DecValTok{4}\NormalTok{)\{}
          \KeywordTok{plot}\NormalTok{(mu[}\DecValTok{1}\NormalTok{,],}
               \DataTypeTok{type=}\StringTok{"o"}\NormalTok{,}
               \DataTypeTok{col=}\StringTok{"blue"}\NormalTok{,}
               \DataTypeTok{ylim=}\KeywordTok{c}\NormalTok{(}\DecValTok{0}\NormalTok{,}\DecValTok{1}\NormalTok{),}
               \DataTypeTok{main =} \StringTok{"K = 4"}\NormalTok{,}
               \DataTypeTok{xlab =} \StringTok{"Number of Dimensions"}\NormalTok{,}
               \DataTypeTok{ylab =} \StringTok{"Estimated Mu"}\NormalTok{)}
          \KeywordTok{points}\NormalTok{(mu[}\DecValTok{2}\NormalTok{,], }\DataTypeTok{type=}\StringTok{"o"}\NormalTok{, }\DataTypeTok{col=}\StringTok{"red"}\NormalTok{)}
          \KeywordTok{points}\NormalTok{(mu[}\DecValTok{3}\NormalTok{,], }\DataTypeTok{type=}\StringTok{"o"}\NormalTok{, }\DataTypeTok{col=}\StringTok{"green"}\NormalTok{)}
          \KeywordTok{points}\NormalTok{(mu[}\DecValTok{4}\NormalTok{,], }\DataTypeTok{type=}\StringTok{"o"}\NormalTok{, }\DataTypeTok{col=}\StringTok{"yellow"}\NormalTok{)}
\NormalTok{        \}}
        \ControlFlowTok{break}\NormalTok{()}
\NormalTok{      \}}
\NormalTok{    \}}
    \CommentTok{#M-step: ML parameter estimation from the data and fractional component assignments}
    \CommentTok{# Your code here}
\NormalTok{    mu =}\StringTok{ }\NormalTok{(}\KeywordTok{t}\NormalTok{(z) }\OperatorTok\StringTok{ }\NormalTok{x) }\OperatorTok{/}\KeywordTok{colSums}\NormalTok{(z)}
    \CommentTok{# N - Total no. of observations}
\NormalTok{    pi =}\StringTok{ }\KeywordTok{colSums}\NormalTok{(z)}\OperatorTok{/}\NormalTok{N}
\NormalTok{  \}}
\NormalTok{  pi}
\NormalTok{  mu}
  \CommentTok{# plot(llik[1:it],}
  \CommentTok{#      type="o",}
  \CommentTok{#      main = "Log Likelihood",}
  \CommentTok{#      xlab = "Number of Iterations",}
  \CommentTok{#      ylab = "Log Likelihood")}
\NormalTok{\}}

\KeywordTok{em_algorithm}\NormalTok{(}\DecValTok{2}\NormalTok{)}
\KeywordTok{em_algorithm}\NormalTok{(}\DecValTok{3}\NormalTok{)}
\KeywordTok{em_algorithm}\NormalTok{(}\DecValTok{4}\NormalTok{)}
\end{Highlighting}
\end{Shaded}


\end{document}
